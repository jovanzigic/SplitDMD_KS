\documentclass[11pt]{article}
\usepackage{graphicx}
\usepackage{amssymb}
\usepackage{multicol}
\usepackage{epstopdf}
\DeclareGraphicsRule{.tif}{png}{.png}{`convert #1 `dirname #1`/`basename #1 .tif`.png}

\textwidth = 6.5 in
\textheight = 9 in
\oddsidemargin = 0.0 in
\evensidemargin = 0.0 in
\topmargin = 0.0 in
\headheight = 0.0 in
\headsep = 0.0 in
\parskip = 0.2in
\parindent = 0.0in

\newtheorem{theorem}{Theorem}
\newtheorem{corollary}[theorem]{Corollary}
\newtheorem{definition}{Definition}

\title{{\tt kuramoto\underline{ }1d.m} - Kuramoto-Sivashinsky Solvers}
\author{Jeff Borggaard}
\begin{document}
\maketitle

\section*{Summary}
This software (and the similar {\tt kuramoto\underline{ }1da.m}) solves the 
Kuramoto-Sivashinsky equation using $C^1$ Hermite cubic elements in a 
periodic domain.

\section*{The Equations}
There are two popular forms of the Kuramoto-Sivashinsky equations.  The first
mimics the form originally derived by Kuramoto \cite{kuramoto}.
%
\begin{equation}
\label{eq:ks1}
 d
\end{equation}

The second has the form of the derivative of equation (\ref{eq:ks1}), namely
%
\begin{displaymath}
  w_t + w_{xx} + w_{xxxx} + 2 w w_x = 0.
\end{displaymath}
%
Note that if we scale the spatial variable so that the computational
domain occurs over the unit interval: $\bar{x} = x/L$, and define
$\epsilon = 1/L^2$, then we have the following version of the Kuramoto-Sivashinsky equation
%
\begin{equation}
\label{eq:ks2}
  w_t + \epsilon w_{\bar{x}\bar{x}} + \epsilon^2 w_{\bar{x}\bar{x}\bar{x}\bar{x}} + 2 \epsilon w w_{\bar{x}} = 0
\end{equation}
%
with $t>0$, $\bar{x} \in (0,1)$, $w(0,t) = w(1,t)$ and $w_{\bar{x}}(0,t) = 
w_{\bar{x}}(1,t)$.  The initial condition is scaled so that 
$w(\bar{x},0) = w_0(\bar{x})$.

A nice property of solutions to equation (\ref{eq:ks2}) is that the integral
%
\begin{equation}
\label{eq:w_ave}
  \int_0^1 w(\bar{x},t) \ d\bar{x}
\end{equation}
%
is independent of time.  In other words, the average of the solutions is
determined from the average of the initial conditions.  This imposes a 
great deal of structure on the solutions as will be seen in the numerical
tests.


\section*{Parameters}
Input parameters
%
\begin{itemize}
  \item {\tt epsilon} - \newline
  This parameter has the effect of changing the length of the periodic
  domain.  See the scaling of the equations above.
\end{itemize}

Internal parameters
%
\begin{itemize}
  \item {\tt n\underline{ }gauss} - \newline
  The number of Gauss point used in element integration
  \item {\tt n\underline{ }nodes} - \newline
  The number of nodes (including the shared endpoints) used in the 
  spatial discretization.  The number of Hermite elements is equal to
  {\tt n\underline{ }nodes-1}.
  \item {\tt t\underline{ }initial}, {\tt t\underline{ }step}, 
  {\tt t\underline{ }save}, and {\tt t\underline{ }final} - \newline
  Time integration parameters.  The solution is saved at multiples of
  {\tt t\underline{ }save}, so {\tt t\underline{ }save}/{\tt t\underline{ }step}
  should be a positive integer.
  \item {\tt resid\underline{ }tol} - \newline
  Residual tolerance for Newton solve at each timestep.
  \item {\tt step\underline{ }tol} - \newline
  Step tolerance for Newton solve at each timestep.
  \item {\tt max\underline{ }iterations} - \newline
  Number of Newton iterations attempted to achieve one of the convergence
  tolerances in each timestep.
\end{itemize}

\section*{Dynamical Systems Analysis}

\begin{table}
\begin{center}
\begin{tabular}{lll}
\hline
length $L$ & \multicolumn{1}{c}{$\epsilon$} & solution behavior \\
\hline
12.5664  &  0.00633257  & bifurcation \\
12.8767  &  0.00603102  & heteroclinic bifurcation \\
13.1403  &  0.00579148  & hopf bifurcation \\
402.2590 &  0.00000618  & ``chaos'' \\
\hline
\end{tabular}
\end{center}
\end{table}

\end{document}